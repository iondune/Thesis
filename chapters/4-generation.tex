
\chapter{Generation}

\begin{figure}
  \centering
    \includegraphics[width=0.8\textwidth]{figures/GeneratorSystem}
  \caption{Terrain generation system overview}
  \label{fig:gen_overview}
\end{figure}

Our terrain generation algorithm creates a world with large continents that contain beaches, forests, hills, and mountains.
The algorithm uses a coarsely sampled world map to determine continent outlines and region boundaries.
For finely sampled terrain values, the values of the world map are used to select and influence custom generators for each region type.
See Figures \ref{fig:gen_overview} and \ref{fig:worldmap}.
The world map is discussed in Section \ref{sec:worldmap} and the different region generators are discussed in \ref{sec:region}

\begin{figure}
	\centering
		\includegraphics[width=1.0\textwidth]{figures/worldmap.png}
	\caption{Example world map section and generator selection.}
	\label{fig:worldmap}
\end{figure}

\section{World Map} \label{sec:worldmap}

The terrain is generated by first establishing a world map at coarse resolution, then selecting from a series of different landscape generators based on general elevation.
The world map uses a single distorted brownian layer with a box ramp to force oceans at all borders.

The base elevations are scaled for a theoretical maximum of 1000 feet and minimum of -750ft.
However, the individual landscape generators may produce values above or below these amounts.

The size and resolution of the world map grid are configurable depending on desired world scale, but our current implementation uses a 2048x2048 world at 512 feet resolution.

\section{Regions} \label{sec:region}

For rendering, the terrain values need to be generated at 1 foot postings.
We also need color values to shade the ground, and a forestation value to determine how to place trees.

Cubic interpolation is used to calculate a baseline elevation at each of these postings.
This elevation is then used to pick a particular generator to generate the final elevation value.
The individual generators are called regions in this system.
The baseline elevation is also used by the region generators to determine how to transition to neighbor regions.

Because the values generated for the world map are sparse (one posting every 1024 feet) it inherently lacks high frequency noise.
This makes it useful for generating features such as cliffs at the intersection between different elevations, because isolines adhere to the grid structure.
In general the use of a low resolution, low frequency baseline map helps keep individual regions separate without having areas that swap between multiple region types unrealistically.
Figure \ref{fig:worldmap} shows an example of the isolines generated by cubically interpolating the world map.

The region types used by the engine are: Oceans, Beaches, Fields, Forests, Hills, Mountains, and Snowpeaks.

While each region generator is individually responsible for generating elevation values, some noise sources are shared between the different regions so that smooth transitions can be generated.
As an example, consider the following odd grouping of region types.

\subsection{Oceans, Fields, Forests}

The oceans, fields, and forests generators all simply add some high-frequency noise and coloring to the baseline elevation.
This generates small sandy hills below water for Oceans, small grassy hills for Fields, and tree-covered hills for Forests.

\subsection{Beaches}

Beaches serve as a transition between oceans and fields.
Beaches may contain a cliff partway between the shoreline and the transition to the field region.

This cliff is generated by applying a vertical offset to any baseline value above a certain threshold.
The offset is applied gradually over a small area so that the cliffs are not completely abrupt, but have a horizontal dimension of a few feet.
As the baseline elevation approaches the transition to Fields, the offset is tapered off to give cliffs some additional prominence.
See Figure~\ref{fig:beach_cliffs}.

The exact size, shape, and shoreline distance of the cliffs are influenced by additional noise values.

\begin{figure}
  \centering
    \includegraphics[width=0.4\textwidth]{figures/beachcliffs}
  \caption{Beach cliff generation}
  \label{fig:beach_cliffs}
\end{figure}

\subsection{Hills, Mountains, Snowpeaks}

Hills, Mountains, and Snowpeaks all use a custom version of a Ridged Multi-Fractal generator.

For the first two octaves of noise, a standard gradient noise sample is used instead of the ridged version.
This helps reduce the sharpness of some peaks and hides some unnatural artifacts that sometimes occur.

In addition, a hilliness parameter is added to help smoothly transition between mountains and hills.
When hilliness is 0, the standard absolute value is used.
At hilliness of 1, the input value is squared instead to produce a parabolic shape.
Hilliness values between 0 and 1 interpolate between these two shapes.

At the low edge of Hill regions, a hilliness value of 1 is used along with low overall amplitude.
As we transition into Mountain and Snowpeak regions, hilliness falls to 0 and amplitude increases.
