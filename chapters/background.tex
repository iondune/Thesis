
\chapter{Background}

\section{Computer Graphics}

Computer Graphics is the process of using computers to produces images.
In realtime applications, this involves using the GPU to rasterize triangles and other primitives.
This process involves producing the geometry to be rendered and then performing a lighting calculation that determines the color of each facet.

\section{Terrain Rendering}

Terrain rendering is a particular subset of computer graphics that is concerned with producing images of terrain and the associated challenges of doing so.
Terrain is important because it plays a large role in the appearance of outdoor scenes, but it is difficult because there is a tremendous amount of geometry.
Terrain rendering methods usually focus on employing level-of-detail algorithms that make it possible to render close-up terrain in high detail and far-away terrain in low detail.
Some of the main challenges these techniques aim to solve are creating the visual transitions between two areas that are rendered at different detail levels, and implementing a scheme to load in new data when the viewpoint changes.

\section{Terrain Generation}

There are many possible sources of terrain data.
Many examples of acquired data are available.
The USGS provides satellite data for much of the world's surface.
Some game engines also use terrain editing tools that allow artists to sculpt terrain.

However, it is also possible to implement algorithms that procedurally construct terrain using a variety of methods.
The advantage of using procedurally generated terrain is that it is less time/effort intensive than artistically creating a terrain, and it does not require much or any disk space (as opposed to using acquired data which can be very large).

Procedural terrain generation methodologies usually employ one or more of the follow techniques.
\begin{itemize}
\item Acquired data utilization
\item Physical process simulations
\item Fractal processes
\item Noise algorithms
\end{itemize}

\subsection{Acquired data utilization}

Some terrain generation processes utilize a small sample of real terrain data to construct a larger terrain surface, or a terrain that is similar to the original input data.
Such processes can be as simple as using texture synthesis to create new patterns, or using noise algorithms to add detail to a low resolution sample.
But some techniques employ more sophisticated processes. \cite{DEMsynthesis}

\subsection{Physical process simulations}

Real terrain is shaped by erosion, the gradual shaping of landscape caued by water, wind, and other forces.
Some terrain generation techniques simulate these physical processes on a starter dataset to create realistic surfaces. \cite{hydrology}

\subsection{Noise Algorithms}

