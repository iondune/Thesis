
\chapter{Related Works}

\section{Geometry Clipmaps: Terrain Rendering Using Nested Regular Grids}

Geometry clipmaps is a good example of a highly scaleable system.
The cost of doubling visible terrain area is low, making it possible to expand the side of visible geometry beyond the practical limitations of floating point numbers. \cite{geometry_clipmaps}

The general approach is to center a regular grid of vertices around the viewpoint, then nest this grid in a larger grid of vertices, and so forth.
Each nested grid has double the resolution of the outer grid such that there is increased detail closer to the viewer.
A torroidally updating texture heightmap is used at each level so that new data can be loaded incrementally.
\editor{Probably need a figure that explains what torroidal updating is}


\section{Terrain Generation Using Procedural Models Based on Hydrology}

This is a terrain generation technique, though the authors also touch on their approach to rendering the terrain surface.
One interesting aspect to note is their novel approach to storing the terrain which focuses on a hierarchy of features instead of only considering geometry.
I think this technique could prove a useful inspiration for designing other scaleable systems besides terrain.

The main point of this paper is to generate terrain using a physically accurate hydrology model.
The results speak for themselves - using their technique produces visually stunning images.


\section{Terrain Synthesis from Digital Elevation Models}

This paper is mostly interesting because of the quality of the resulting terrains.
Their technique involves sampling real terrain data using a user-controlled shape/template to generate a new terrain.

For some games this process has tremendous potential for allowing some artist control of the surface while still importing tremendous detail that doesn't need to be artistically controlled.


\section{Effective Water Simulation from Physical Models}


\section{Interactive Landscape Visualization Using GPU Ray Casting}


\section{Real-time Realistic Rendering and Lighting of Forests}


\section{Real-time Realistic Ocean Lighting using Seamless Transitions from Geometry to BRDF}


\section{Precomputed Atmospheric Scattering}


\section{Real-time rendering and editing of vector-based terrains}


\section{The Synthesis and Rendering of Eroded Fractal Terrains}


