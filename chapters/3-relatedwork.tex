
\chapter{Related Works} \label{rworks}

\section{Terrain Generation}

Some recent publications in the field of terrain generation have produced impressive landscapes.
G{\'e}nevaux et al. used a vector-based hydrology simulation with user-controllable parameters \cite{hydrology}.
One interesting aspect to note is their novel approach to storing the terrain which focuses on a hierarchy of features, similar to constructive solid geometry.
Zhou et al. implemented a system to stitch together samples from real-world DEM data, also with user-controllable parameters \cite{DEMsynthesis}.

By comparison, the generation system Relic uses is much simpler, but it allows height values to be easily generated at arbitrary resolution.

\section{Tree Rendering}

Many published techniques for rendering forests involve ray casting or ray marching.
Mantler et al. add tree rendering to a terrain renderer by ray marching on an appended tree heightmap \cite{terraintreecast}.

Bruenton \& Neyret use pre-rendered depth maps of tree models from many possible angles in addition to an algorithm for distant forest shading \cite{bruneton_trees}.
Pre-rendered depth maps are a more complex version of impostor rendering.
The low-poly trees of Relic are more sufficiently represented by an impostor than photo-realistic trees, and impostors are far less intensive to both generate and render than depth maps.

\section{Minecraft}

{\em Minecraft} is an open-world game that uses voxels for representing and rendering terrain \cite{word_of_notch}.
The voxel representation makes it possible to support caves and overhangs, as well as real-time editing.
However, all terrain is represented at the same detail level, making it difficult to support large view distances.
Additionally, {\em Minecraft} has a vertical range limited to 256 meters \cite{minecraft_altitude}.

\section{Voxels and Volumetric Terrain}

Marching cubes is an algorithm for creating a polygonal mesh from volumetric data \cite{marchingcubes}.
Unlike the voxel meshing used by {\em Minecraft}, Marching cubes can be used to create smooth surfaces because grid values are scalar distances, not boolean occupancy values.
Relic's voxel algorithm operates on a boolean grid (which is more friendly for problems like in-game editing) but can still produce slanted surfaces.

The Transvoxel Algorithm is a technique for stitching together volumetric meshes so that level-of-detail techniques can be applied to volumetric terrain \cite{lengyel}.
Relic avoids the need to apply LOD to volumetric terrain by using a heightmap representation for distant terrain.



% Effective Water Simulation from Physical Models
% Real-time Realistic Ocean Lighting using Seamless Transitions from Geometry to BRDF
% Precomputed Atmospheric Scattering


% The Synthesis and Rendering of Eroded Fractal Terrains



