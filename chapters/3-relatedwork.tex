
\chapter{Related Works} \label{rworks}

\section{Terrain Generation}

Some recent publications in the field of terrain generation have produced impressive landscapes.
G{\'e}nevaux et al. used a vector-based hydrology simulation with user-controllable parameters. \cite{hydrology}
One interesting aspect to note is their novel approach to storing the terrain which focuses on a hierarchy of features, similar to constructive solid geometry.
Zhou et al. implemented a system to stitch together samples from real-world DEM data, also with user-controllable parameters. \cite{DEMsynthesis}

\section{Tree Rendering}

Many published techniques for rendering forests involve ray casting or ray marching.
Mantler et al. add tree rendering to a terrain renderer by ray marching on an appended tree heightmap. \cite{terraintreecast}
Bruenton \& Neyret use pre-rendered depth maps of tree models from many possible angles in addition to an algorithm for distant forest shading. \cite{bruneton_trees}

% \section{Minecraft}





% Effective Water Simulation from Physical Models
% Real-time Realistic Ocean Lighting using Seamless Transitions from Geometry to BRDF
% Precomputed Atmospheric Scattering


% The Synthesis and Rendering of Eroded Fractal Terrains



