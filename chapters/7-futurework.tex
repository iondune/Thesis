
\iffalse \bibliography{../bibliography.bib} \fi

\chapter{Future Work}

\section{Terrain Lighting}

\subsection{Shadows}

Besides the HBAO+ post-processing pass applied to the whole scene, no shadows are currently calculated.
The scene is inherently difficult to shadow due to the large view scales.
A cascaded shadow map system with robust view frustum culling could be applied to the geometry clipmap layers but the performance hit on rendering may be significant.

One option to improve rendering performance is horizon-based shadowing.
Horizon-base shadowing pre-calculates horizon values for each pixel in a heightmap.
These horizon values are then compared against sun elevation to determine whether a fragment is in shadow.

While calculating horizon values for all clipmap layers may be expensive, it might be possible to calculate horizon values for e.g. every fourth layer and re-use the low resolution horizontal values in high resolution layers.


\subsection{Accurate Scattering}

Our current scattering system is a very rough approximation of real scattering.
There are open source solutions with more accurate simulations of the Rayleigh and Mie scattering which could be used.

It might also be better to use a non-photo-realistic color-ramp scattering simulation such as the one used by Firewatch.


\subsection{Volumetric Lighting}

Nvidia Gameworks, which our system uses for screen-space ambient occlusion, also has an implementation of volumetric lighting that could be used for more accurate sunset and sunrise effects.


\section{Water}

Our water system uses the shape of geometry clipmaps for simplicity but this causes a lot of water vertices to be rendered off-screen.
Using a grid projected from screen-space is a common way to render water at different view scales.


\section{Terrain Generation}

Our system uses a standard grid to store the world map, but this brings some square artifacts into the large-scale terrain features.
Other similar systems use a Voronoi diagram to produce more varied large-scale structure.


\section{Vegetation}

While the mesh facade rendering system is cheap, it is a very rough visual approximation of distance forests.
It is also not sufficiently cheap to render at significant scale.
As such, the current system has a much shorter view distance for vegetation than it does for terrain.

Using a GPU ray-cast simulation of forests could produce much larger view distances and improved visual approximation. \cite{terraintreecast}
