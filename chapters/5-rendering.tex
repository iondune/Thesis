
\chapter{Rendering}

\section{Nearby Terrain}

Nearby terrain is rendering using a pseudo-voxel representation designed to allow for sloped surfaces while maintaining intuitive live editing capabilities.
A voxel algorithm that uses uniform voxel size is simple to implement and manage but is poorly suited to represent sloped surfaces.
Fully volumetric terrain implementations allow for arbitrary slopes but are less intuitive for manipulation by players and navigation by AI.
We utilize a pseudo-voxel system that allows for diagonal faces in addition to solid voxels.
This system allows for simple grid-based editing and simple physics calculations, while allowing sloped polygonal faces that are more aesthetically pleasing than simple voxels.

\subsection{Voxel algorithm}

In principle, the system works by breaking down each individual voxel into 8 sub-voxels (one for each corner of the cube) and generating diagonal faces based on which of these sub-voxels are occupied.
However, certain sub-voxel configurations are considered degenerate and are automatically trimmed to a lesser configuration.
For example, in our system any pseudo-voxel with only a single sub-voxel occupied is automatically trimmed to an empty voxel.
In this case the trimming occurs because there is no diagonal face to represent just a single corner of a voxel.
Figure \editor{needed} shows the pseudo-voxel with the fewest possible sub-voxels, four.
The system can therefore trivially trim any pseudo-voxel configuration with fewer than four sub-voxels.

The system also trims certain pseudo-voxel configurations when the resulting triangle faces would simply confusing.
See Figure \editor{needed}.

\subsection{Face Generation}

Triangles are generated for each pseudo-voxel configuration using lookup tables for efficiency and simplicity.
Triangles are divided into two categories: interior and exterior.

\subsubsection{Exterior Faces}

Exterior triangles are the triangles that would normally be generated by a simple voxel system - the faces of a cube.
Exterior triangles are checked against each neighboring pseudo-voxel for visibility.
In the trivial case, a completely solid voxel surrounded entirely by solid voxels produces no triangles, since all exterior triangles are occluded by the exterior faces of each neighboring voxel.

Note that while a normal voxel system can arbitrarily decide how to triangulate the quad faces of a cube, our system must support either possible triangulation in order to match up with all possible diagonal faces.
See Figure \editor{needed} for an example of why arbitrary quad triangulation is needed.

\subsubsection{Interior faces}

Interior faces are what make the pseudo-voxel system interesting - any diagonal face that caps a partial pseudo-voxel.
See Figure \editor{needed} for a table of all interior faces.

\subsection{Rendering}

The world is divided into cube chunks for which a triangle mesh is generated.
Each chunk tracks its neighbors in all six face directions so that occluded exterior faces can always be accurately detected.
This means that a ring of non-visible chunks must be loaded around all visible chunks, since no visible chunk can have an unloaded neighbor.

Screen-space ambient occlusion is used to help visual understanding of the voxel terrain shape.


\section{Far Terrain}

Far terrain is rendered using an implementation of geometry clipmaps with a few modifications.

Geometry shader normal calculation.


\section{Vegetation}

Mesh instances.

Impostors.


\section{Water}

Geometry clipmaps.

Gerstner waves.
